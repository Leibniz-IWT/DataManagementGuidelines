\section{Dateiorganisation}\label{ssc:Dateiorganisation}
Im Folgenden wird erläutert, welche Dateien Sie wie auf unseren Servern
speichern:
\begin{itemize}
  \item Der Zugang zum VT-Server erfolgt über Ihren Webbrowser durch Eingabe der Webadresse
        in das Adressfeld, über die Zuordnung des VT-Servers zum
        Windows-Explorer als eigenes Laufwerk, oder über den Synology Drive
        Client (siehe \cite{synology2022}) mit einigen weiteren und sehr
        hilfreichen Optionen wie z.B. der Versionierung von Dokumenten.
        Die \textbf{maximale Größe} der hochgeladenen Daten sollte
        \textbf{100Gbyte pro Datei} nicht überschreiten. Die
        Zugriffsberechtigung wird von
        Stefan Endres (\href{mailto:s.endres@iwt.uni-bremen.de}%
                       {s.endres@iwt.uni-bremen.de}),
        Arvind Chouhan (\href{mailto:a.chouhan@iwt.uni-bremen.de}%
                        {a.chouhan@iwt.uni-bremen.de})
        und Nils Ellendt (\href{mailto:ellendt@iwt.uni-bremen.de}%
                          {ellendt@iwt.uni-bremen.de})
        verwaltet.
  \item Der Server der Abteilung Fertigungstechnik ist ebenfalls mittels
        Browser oder Windows-Explorer erreichbar. Die Zugriffsberechtigung wird von der
        IT-Abteilung des IWTs verwaltet.
\end{itemize}

\subsection{Verschiedene Datentypen}
Je nach Herkunft der Daten wird zwischen zwei Datentypen unterschieden \cite{gitelman2013}:
\begin{itemize}
  \item Primärdaten (manchmal auch Rohdaten genannt) werden von Messgeräten,
        Sensoren/Kameras, Simulationsprogrammen, Beobachtungen usw. erzeugt
        und sind unbearbeitet; sie werden von Forschenden zum ersten Mal erzeugt.
  \item Sekundärdaten (oder analysierte Daten) sind Daten, die aus der
        Verarbeitung von Primärdaten hervorgehen. (In anderen
        Wissenschaftsbereichen sind Sekundärdaten bereits von anderen
        bearbeiteten oder erzeugten Daten).
\end{itemize}
Sehr große Primärdaten können nicht auf unserem Electronischen Labor-Notizbuch (ELN)%
  \footnote{%
    Die Nutzung des ELN ist derzeit nur für die VT verpflichtend.
  }%
gespeichert werden, aber die Herkunft dieser Daten wird dort zusammen mit
Informationen über den Speicherort (z.B. wo genau sie auf einem Dateiserver
liegen) erklärt. Dabei gelten folgende Regeln:
\begin{itemize}
  \item Daten $<$ 100 MByte werden im ELN gespeichert und dokumentiert
  \item 100 MByte $<$ Daten $<$ 1TByte werden auf dem VT-Server gespeichert und im
        ELN dokumentiert
  \item Daten $>$ 1 TByte werden auf USB-Disks gespeichert und im ELN dokumentiert
\end{itemize}
Es ist unbedingt erforderlich, dass \textbf{alle Daten aus externen Quellen} (z.B. von einem USB-Stick) manuell auf Viren geprüft werden müssen, wenn der automatische Scan nicht funktioniert, bevor sie auf der Festplatte verwendet/gespeichert werden.

\subsection{Benennung von Dateien und Ordnern }

Nehmen Sie sich zu Beginn eines Projekts Zeit für die Festlegung von
Dateibenennungen. Wie werden Sie oder andere später nach Dateien suchen und auf
sie zugreifen? Denken Sie dabei an den Typ, den Speicherort, die Studie oder
ähnliches.
\begin{itemize}
  \item Dateinamen sind selbsterklärend – sie beschreiben die Daten,
        um die es sich handelt!
  \item Verwenden Sie keine Leerzeichen oder Sonderzeichen, sondern nutzen Sie
        CamelCase, Bindestrich ‘-’ oder Unterstich ‘\_’ als Trennzeichen: \\
        \begin{tabular}{ll}
          \textcolor{red}{\textbf{NO}}  & \textcolor{red}{‘name date v1.txt’} \\
          \textbf{YES} & ‘name\_date\_v01.txt’ \\
          \textbf{YES} & ‘nameDateV01.txt’ \\
          \textbf{YES} & ‘simpleFoam\_u0.2m-s\_etha2mPas\_v01’ \\
%          \textcolor{green}{\textbf{YES}} & \textcolor{green}{‘name\_date\_v01.txt’} \\
%          \textcolor{green}{\textbf{YES}} & \textcolor{green}{‘nameDateV01.txt’} \\
%          \textcolor{green}{\textbf{YES}} & \textcolor{green}{‘simpleFoam\_u0.2m-s\_etha2mPas\_v01’} \\
        \end{tabular}
  \item Entwickeln Sie ein Dateibenennungsschema, das Informationen über die
        Daten enthält. Beispiel: \\
        {[Date]}\_{[Run]}\_{[SampleType]}
  \item Berücksichtigen Sie die Sortierung, wenn Sie entscheiden, welches
        Element des Dateinamens an erster Stelle steht: \\
        Verwenden Sie ‘JJJJ-MM-TT‘, um Daten zu speichern. \\
        Verwenden Sie führende Nullen: \\
        \begin{tabular}{lll}
          \textcolor{red}{\textbf{NO}}  & \textcolor{red}{‘ProjID\_v1.csv’  	...  	‘ProjID\_v11.csv’} & \\
          \textbf{YES} & ‘ProjID\_v01.csv’ 	... 	‘ProjID\_v11.csv’
             & (Reihenfolge bis 99) \\
          \textbf{YES} & ‘ProjID\_v001.csv’ 	...	‘ProjID\_v111.csv’
             & (Reihenfolge bis 999) \\
        \end{tabular}
\end{itemize}

\subsection{Dateiformate}
Aufgrund des technologischen Wandels veralten sowohl Hardware als auch Software.
Deshalb müssen die von Ihnen gewählten Dateiformate eine langfristige Lesbarkeit
bzw. Zugriff gewährleisten.\\
Dateiformate, die auch in Zukunft zugänglich sein werden, haben die folgenden
Eigenschaften:
\begin{itemize}
  \item Nicht-proprietär
  \item Offener, dokumentierter Standard
  \item Gemeinsame Nutzung durch die Forschungsgemeinschaft
  \item Standarddarstellung (ASCII, Unicode)
  \item Unverschlüsselt
  \item Unkomprimiert (wenn nicht zu groß...)
\end{itemize}

\noindent Beispiele für \textbf{bevorzugte} Dateiformate sind:
\begin{itemize}
  \item Offene Dokumentdateien: .odt (Text), .ods (Tabellenkalkulation), .odp
        (Präsentation); primäres Format von OpenOffice/Libre Office,
        standardisiert in ISO 26300
  \item Office Open XML: .docx (Text), .xlsx (Tabellenkalkulation), .pptx
        (Präsentation); primäres Format von Microsoft Office, genormt in ISO 29500
  \item Ritch Text Format: .rtf; proprietäres, aber gut dokumentiertes Format
  \item ASCII-Text oder Auszeichnungssprache: .txt, .md (Markdown), .html
        (Hypertext), .tex (LaTeX), .csv (Komma-getrennte Werte)
  \item Portable Document Format: .pdf; genormt in ISO 32000
  \item Bibliografische Daten: .enl (EndNote; Standardsoftware für die
        Literaturverwaltung am IWT); ansonsten verwenden Sie bitte lesbare
        Formate wie .bib (BibTeX) oder .ris (Research Information System Format)
  \item Videodateien in den Datenformaten .mp4 und .mpg, die mit den Codecs
        H.264 und H.265 erzeugt wurden; standardisiertes Format (ISO 14496-10)
        für die Videokomprimierung; verwenden Sie bitte \textbf{keine
        proprietären Codecs wie etwa Quicktime (.mov, .qt)}
  \item Bilder: .jpg ist das Standard-Dateiformat der meisten Kameras. Es kann
        von fast allen Bildbearbeitungsprogrammen gelesen werden, ist aber nicht
        in allen Fällen ideal (z.B. Strichzeichnungen), weil es ein
        verlustbehaftetes Format ist, das je nach gewählter Komprimierung zu
        einer dauerhaften Verschlechterung des Bildes führen kann. Das
        Nachfolgeformat .jp2 (JPEG2000) ermöglich eine verlustfreie
        Komprimierung und kann daher für die Bildarchivierung verwendet werden,
        ist aber aufgrund Lizenzproblemen nicht weit verbreitet. Das
        "Tagged Image File Format" .tiff ist ein wichtiges Format für den
        Dateiaustausch mit Verlagen. Es ermöglicht eine verlustfreie
        Komprimierung und eine hohe Farbtiefe (32 Bit) sowie CMYK-Farben.
        Es kann für die Bildarchivierung verwendet werden, allerdings bleibt die
        Dateigröße vergleichsweise groß. Das " Portable Network Graphic" Format
        .png ist ein Rasterbildformat mit guter verlustfreier Komprimierung.
        Es ergibt größere Dateigrößen als .jpg für Fotos, ist aber sehr gut für
        Strichzeichnungen geeignet.
  \item Für Skizzen und Zeichnungen sind vektorbasierte Dateiformate
        vorzuziehen, da diese eine unbegrenzte Skalierung des Bildes ohne
        Qualitätseinbußen ermöglichen. Bei großen Ausdrucken sind
        Vektorgrafiken deutlich kleiner als Rastergrafiken. Allerdings werden
        vektorbasierte Bilder nicht von allen Textverarbeitungsprogrammen gut
        verarbeitet, und selbst wenn sie verwendbar sind, können sie auf
        verschiedenen Computern unterschiedliche Ergebnisse liefern. Es wird
        empfohlen, eine Vektorgrafik in einem offenen Format (.svg, scalable
        vector graphics) zu erstellen, um eine langfristige Lesbarkeit zu
        gewährleisten, und sie in das für Ihr Textverarbeitungsprogramm am
        besten geeignete Dateiformat zu exportieren (z. B. .emf für Word, .eps
        oder .pdf für LaTeX).
  \item Binäre Daten: Manchmal ist es am effizientesten, Daten in einem binären
        Format zu speichern, da textbasierte Darstellungen die Dateigröße
        drastisch erhöhen würden. In diesem Fall ist es wichtig, zu
        dokumentieren, wie die Daten gespeichert werden, und vorzugsweise ein
        Beispiel dafür zu geben, wie sie wieder gelesen werden können.
\end{itemize}

\noindent Wenn Sie in Erwägung ziehen, Ihre (Mess-)Daten in ein Format mit den
oben genannten Eigenschaften zu exportieren, bewahren Sie eine Kopie im
ursprünglichen Softwareformat auf, es sei denn, Sie können sicherstellen, dass
alle Daten und Metadaten korrekt konvertiert wurden. Wenn Sie Ihre Daten in
einem Repository hinterlegen, können Ihre Dateien in neuere Formate migriert
werden, sodass sie auch für zukünftige Forschende nutzbar sind.

\subsection{Ordnerhierarchie}

Der spezifische Wert von Daten hat eine große Streuung. Viele Messungen,
z.B. aus Vorversuchen, waren evtl. für die Planung Ihrer Experimente sehr
wichtig, finden aber keine Verwendung in Veröffentlichungen. Im Gegensatz dazu
sind einige Ihrer Daten sehr wertvoll und finden ihren Weg in eine
Veröffentlichung. Aufgrund dieser Unterschiede hat der Ort, an dem Sie Ihre
Daten speichern, je nach diesen drei Datenarten eine unterschiedliche
Wertigkeit:
\begin{enumerate}[label=\Roman*.]
  \item Daten mit einer geringen Wertigkeit, siehe:
       \ref{ssc:daily-data} \nameref{ssc:daily-data}
  \item Daten, die wahrscheinlich für eine Veröffentlichung bestimmt sind,
        siehe:
        \ref{ssc:data-for-publication} \nameref{ssc:data-for-publication}
  \item Daten von akzeptierten Publikationen: Diese Daten werden in einem
        speziellen Bereich des File-Servers gespeichert. in den Sie Ihre Daten nur
        einmal schreiben können – Sie können dort nichts ändern oder löschen.
        Es ist wie ein Briefkasten: Sobald Sie Ihre Steuererklärung eingeworfen
        haben, ist es vorbei. Dieser Datenbereich ist nur für akzeptierte
        Veröffentlichungen gedacht, bei denen wirklich alles in Ordnung ist!
\end{enumerate}

\subsubsection{Tägliche Daten}\label{ssc:daily-data}

Jeden Tag generieren Sie Daten z.B. durch Messungen, Simulationen, Erstellung
eines Papers usw. Dafür sind die NAS-File-Server vorgesehen. All diese täglichen
Daten werden unter \texttt{Mitarbeiter} und \texttt{<Familienname>} gespeichert,
z.B.: \\
\texttt{/VT-Server/<Abteilung>/Mitarbeiter/<Familienname>/} \\
Die von \texttt{<Abteilung>} eingeschlossenen Namen müssen aus unseren
Abteilungen gewählt werden, z.B.:
\begin{itemize}
  \item LFM (Labor für Mikrozerspanung)
  \item MPS (MehrPhasenStrömung)
  \item RST (Reaktive Sprüh-Technik)
  \item SM (Strukturmechanik)
  \item SPK (SprühKompaktieren)
\end{itemize}
je nach Ihrer Zugehörigkeit. Ein Beispielverzeichnis ist: \\
\texttt{/VT-Server/MPS/Mitarbeiter/riefler/} \\
\\
Hier speichern Sie die folgenden Daten:
\begin{itemize}
  \item[$\rightarrow$] \textbf{Experimentelle Daten:}
    \begin{itemize}
      \item Alle Experimente des Projekts
      \item Zusammenhang zwischen diesen Experimenten (welche Parameter wurden
            variiert) in einer JSON (Java Scipt Object Notation)-basierten Textdatei, siehe
            \ref{sc:data-documentation} \nameref{sc:data-documentation} weiter unten
      \item Zwischenergebnisse
      \item Liste aller verwendeten Geräte mit Einstellparametern
    \end{itemize}
  \item[$\rightarrow$] \textbf{Experimentelle Aufbauten:}
    \begin{itemize}
      \item Skizzen aller Teile
      \item Liste aller benötigten Teile und Geräte
    \end{itemize}
  \item[$\rightarrow$] \textbf{Für Simulationsdaten:}
    \begin{itemize}
      \item Die erforderlichen Anfangswerte und Netzinformationen, um alle
            Simulationen des Projekts neu zu berechnen
      \item Relation zwischen diesen Simulationen (welche Parameter variiert
            werden) in einer JSON-basierten Textdatei, siehe
            \ref{sc:data-documentation} \nameref{sc:data-documentation} weiter unten
      \item Zwischenergebnisse
      \item Liste der verwendeten Software zusammen mit dem verwendeten Quellcode
      \item Alle erforderlichen Eingabedateien zur Reproduktion der Simulationen
    \end{itemize}
\end{itemize}

\noindent Es gibt keine vorgegebene Verzeichnisstruktur für Ihre täglichen
Daten, das bleibt Ihnen überlassen. Diese Daten unter Ihrem Namen unterscheiden
sich jedoch von Projektdaten wie z.B. wichtigen experimentellen Ergebnissen oder
Simulationen, die veröffentlicht werden und die im Folgenden beschrieben werden.

\subsubsection{Daten zur Veröffentlichung}\label{ssc:data-for-publication}

Hier speichern Sie alle Daten, die für jede Veröffentlichung innerhalb dieses
Projekts relevant sind. Abhängig von der Größe der Daten und der Entscheidung
innerhalb Ihrer Abteilung (siehe oben) dürfen Primärdatendateien, die größer als
z.B. 10 GByte sind, nicht in dem hier angegebenen Verzeichnis gespeichert
werden. Stattdessen muss innerhalb der Dokumentation in der
Experimentbeschreibung im ELN ein Hinweis auf die entsprechende Herkunft (z.B.
Hochgeschwindigkeitskameramessung oder CFD-Simulation) gegeben werden und wo
genau die Primärdaten gespeichert sind. Das \textbf{entscheidende Grundprinzip}
für alle Ihre Roh- und Primärdaten ist: \textbf{\underline{Alle Daten müssen in
ihrer Herkunft nachvollziehbar dokumentiert sein!}}

Wenn Sie ein Dokument vorbereiten, z.B. eine Präsentation oder einen
Projektbericht, werden die Daten (nach einem strengen Schema, siehe \ref{app:dummy-paper} \nameref{app:dummy-paper}) als Beispiel in diesem Verzeichnis gespeichert: \\
\texttt{/VT-Server/<Abteilung>/Projekte/<Finanzierung>/<Projektname>}
\begin{itemize}
  \item[$\rightarrow$] \textbf{Ergebnisdokumentation:}
    \begin{itemize}
      \item Dokumente aus Ihrer wissenschaftlichen Arbeit: Papers,
            Dissertationen, Präsentationen, Berichte, Proposals, etc.
      \item Alle Ergebnisse (Abbildungen, Tabellen, Filme, Animationen, ...)
            und die zugrundeliegenden gemessenen oder simulierten Daten
            zusammen mit Auswertungsprogrammen
    \end{itemize}
\end{itemize}
Finanzierungsquellen (\texttt{<Finanzierung>}) sind z.B.: DFG, AiF, BMBF, Industrie, ERC, etc. Zuletzt wird der Verzeichnisname für veröffentlichte Arbeiten in einem Projekt zusammengefügt aus diesen Komponenten: \\
\texttt{<Jahr Projektstart>\_<Abteilung>\_<Projektbezeichnung>\_<Familienname> }.
Siehe \ref{app:dummy-paper} für ein Beispiel sowie die beiden hier angeführten
Beispiele:
\begin{itemize}
  \item \texttt{/VT-Server/MPS/Projekte/DFG/2016-MPS-Tropfengenerator\_Riefler/...}
  \item \texttt{/VT-Server/RST/Projekte/DFG/2018\_RST\_Flatbandpotential\_Naatz/...}
\end{itemize}
Die Dopplung von MPS und RST ist auf einen gemeinsamen Dateiserver
zurückzuführen, auf dem in Zukunft die Daten aller IWT-Abteilungen
gespeichert werden sollen.

\paragraph{Papers und Dissertationen}

\noindent In Absprache mit allen IWT-Abteilungen basiert das
Verzeichnisbenennungsschema für veröffentlichte Daten des VT-Servers auf den
Abteilungen, der Finanzierungsquelle und dann dem entsprechenden Projektnamen
(siehe oben für die Definition von \texttt{<Projektname>}): \\
Texte und Daten eines Papers werden z.B. in diesem Verzeichnis gespeichert:
\\
Texte und Daten eines Papers werden z.B. in einem Verzeichnis gespeichert:  \\
\texttt{/VT-Server/<Abteilung>/Projekte/<Förderung>/<Projektname>/Paper/} \\
mit diesem Verzeichnisnamen:
\\
\texttt{<Jahr>\_<Abteilung>\_<Paper-Name>\_<Erstautor> } \\
Texte und die Daten einer Dissertation werden dagegen so
abgelegt:
\\
\texttt{/VT-Server/<department>/Projekte/<funding>/<project name>/Dissertation/} \\
Mit einem Verzeichnisnamen wie:
\\
\texttt{<Jahr>\_<Abteilung>\_<Dissertationstitel>\_<Autor>} \\
In diesen Verzeichnissen speichern Sie alle Daten Ihrer Arbeit in
Unterverzeichnissen, deren Namen in der linken Zeile in
Tabelle \ref{table:paper-directory-structure} angegeben sind. Die Namen
der Unterverzeichnisse sind immer alle auf Englisch (wegen der Eindeutigkeit)
und in der linken Zeile aufgeführt, die zugehörigen Dateien werden rechts
erklärt:
\begin{table}[!h]
 \begin{tabularx}{\linewidth}{l|p{1mm}X}
  \toprule
  \midrule
  \multirow{2}{*}[-17pt]{\texttt{00\_FinalPublication}$^*$} &
    \,\tabitem & The very last version of your paper as a pdf with volume, year
               and page numbers on the VT-Server into \texttt{’VT-Publikationen’} \\
    & \,\tabitem & EndNote entry for IWT-WiKo including DOI number, and an
                   EndNote *.ris ASCII export fulfilling the FAIR principles \\
  \midrule
  \texttt{01\_Manuscript} &
    \,\tabitem & the *.pdf and *.docx or *.tex \\
  \midrule
  \multirow{1}{*}[-25pt]{\texttt{02\_Figures}$^{**}$} &
    \,\tabitem & one sub-directory for every figure (\texttt{figure\_01}, ...)
                 with data points as *.csv or *.txt, or the generating
                 file/program for the figure points (Matlab, Python,
                 Origin,...); the results are from data stored in or extracted
                 from the \texttt{04\_PrimaryData} directory \\
  \midrule
  \multirow{1}{*}[-7pt]{\texttt{03\_Tables}$^{**}$} &
    \,\tabitem & one sub-directory for every table (\texttt{table\_01}, ...) with
                 data stored in as a csv-file \\
  \midrule
  \multirow{2}{*}[-8pt]{\texttt{04\_PrimaryData}} &
    \,\tabitem & measured or simulated data, each sub-directory with a
                 `readme.json' file describing the data \\
    & \,\tabitem & ELN pdf pages with QR-code (see below) \\
  \midrule
  \multirow{2}{*}{\texttt{05\_ProgramSources}} &
    \,\tabitem & e.g. OpenFOAM-Solver, Python/Matlab code, etc. \\
    & \,\tabitem & case files (mesh) for OpenFOAM, Fluent, ABAQUS etc. \\
  \midrule
  \multirow{2}{*}{\texttt{06\_References}} &
    \,\tabitem & the references *.bib or *.enl together with *.ris  \\
    & \,\tabitem & Cited papers/books (if publication rights are permitted) \\
  \midrule
  \multirow{1}{*}{\texttt{07\_Supplement}} &
    \,\tabitem & Supplementary files  \\
  \midrule
  \multirow{1}{*}{\texttt{08\_Misc}} &
    \,\tabitem & Reviewer comments / rebuttals  \\
  \midrule
  \bottomrule
\end{tabularx}

  \caption{%
  Die Daten jeder Arbeit müssen in acht Unterverzeichnissen gespeichert werden;
  weitere Hinweise: \\
  **Wenn die Arbeit druckreif ist, lädt der korrespondierende Autor die letzte
  und überarbeitete Version (d.h. nur die tatsächlich verwendeten, aber absolut
  vollständigen Daten) in das geschützte Verzeichnis \\
  ***Wenn die in einer Abbildung/Tabelle wiedergegebenen Daten aus verteilten
  Primärdatenverzeichnissen stammen, genügt es, hier nur das erzeugende Programm
  zu speichern%
  }
\label{table:paper-directory-structure}
\end{table}

Im Fall einer Dissertation speichern Sie Ihre Präsentation der Promotionsprüfung zusammen mit dem Manuskript unter dem Verzeichnis \texttt{"01\_Manuscript+Presentation"}. Wenn Sie keine ergänzenden Informationen zu Ihrer Publikation/Dissertation haben oder keine spezifischen Computerprogramme für Ihre Forschung geschrieben haben, müssen die Verzeichnisse \linebreak
\texttt{"05\_ProgramSources"}, \texttt{"07\_Supplement"} und \texttt{"08\_Misc"} noch vorhanden sein, aber ohne Einträge. Die Änderungen am Manuskript aufgrund des Revision-Prozesses können durch Anlegen von Unterverzeichnissen \texttt{"01\_Manuscript/01\_Submission"} sowie \texttt{"01\_Manuscript/02\_Revision"} samt Speichern der jeweiligen Manuskript- und Rebbutal-Datei nachvollziehbar gemacht werden.

Im Verzeichnis \texttt{"02\_Figures"} speichern Sie jede Abbildung aus dem Manuskript in guter Qualität und zusätzlich eine *.csv-Datei mit den Daten der Abbildung. Oder Sie speichern eine Skript-Datei, die aus den Primärdaten im Verzeichnis \texttt{"04\_PrimaryData"} die entsprechende Abbildung generiert - zum Beispiel eine Matlab-Datei. Dieses Datenverzeichnis enthält die gemessenen oder simulierten Daten im *.txt- oder *.csv-Format aus dem Messgerät oder der Simulationssoftware und eine weiter unten (\ref{sc:data-documentation}  \nameref{sc:data-documentation}) beschriebenen, z.B. vom Readme-File-Creator erzeugte "readme.json"-Datei mit Informationen und Metadaten zu Ihren gemessenen / simulierten Daten. Der Bezug jedes Datenpunktes in einer Abbildung zu den Primärdaten muss entweder implizit, z.B. in der Matlab-Datei, oder besser explizit in der "readme.json"-Datei angegeben werden.

Um Ihre Veröffentlichung zu vervollständigen, sollte jede zitierte Arbeit in das Verzeichnis \texttt{"06\_References"} für internen Gebrauch aufgenommen werden. Allzu oft können die Ideen einer Arbeit nicht nachvollzogen und reproduziert werden, weil es Referenzen gibt, die Sie nicht bekommen können, weil Ihre
Institution keinen Zugang zu dieser speziellen Zeitschrift bietet, oder die Referenzen stammen aus einer Masterarbeit, die nicht verfügbar ist. Speichern Sie daher alle Referenzen mit geeigneten Dateinamen für einen direkten Bezug - zusammen mit allen zuvor genannten Daten und Informationen - das stellt Ihre Arbeit in ihrer Gesamtheit dar. Darüber hinaus ermöglicht EndNote eine komfortable Direktverlinkung zu den zitierten Publikationen, d.h. Sie sollten diese Links in Ihrer '*.enl'-Datei hinzufügen. Das Gleiche ist auch in BibTeX
möglich. Wenn Ihre Arbeit auf ein öffentliches Repository übertragen werden soll, können jedoch nur frei zugängliche Publikationen hochgeladen werden, da viele Verlage, wie Springer und Elsevier, es nicht erlauben, Veröffentlichungen aus ihren Zeitschriften oder Büchern so zu veröffentlichen.

Das Verfassen einer Arbeit ist ein mehrstufiger Prozess. Mit der Entscheidung für eine bestimmte Zeitschrift kommt der Prozess an einen entscheidenden Punkt, an dem alle Daten in der oben und im folgenden Screenshot angegebenen Struktur gespeichert werden müssen und der Begutachtungsprozess beginnt. Wird der Beitrag
angenommen, werden alle Änderungen integriert und das endgültige pdf mit Band, Jahr und Seitenzahlen freigegeben; die letzten Aktionen sind dann:
\begin{itemize}
  \item Prüfen Sie, ob die wirklich benötigten Daten gespeichert sind, nicht
        mehr und nicht weniger.
  \item Überprüfen Sie die Übereinstimmung der Nummerierung der Abbildungen und
        Tabellen im Manuskript mit den entsprechenden Verzeichnissen und dem
        Erzeugungsprogramm der einzelnen Abbildungen (sehen Sie sich das
        Manuskript im DummyPaper an). Tabellen ohne Bezug zu Mess- oder Simulationsdaten
        müssen nicht gesondert im Verzeichnis \texttt{"03\_Tables"} abgelegt werden.
  \item Prüfen Sie, ob alle ELN-Dateien, die das Experiment/die Simulation
        beschreiben, abgelegt wurden.
  \item Allerletzter Schritt: Legen Sie eine Kopie der kompletten Daten des
        Papers in das Verzeichnis 'Published' auf dem Fileserver. Das ist ein
        spezielles Verzeichnis: Sie können etwas dorthin kopieren, aber danach
        können Sie nichts mehr ändern. Es ist wie ein Briefkasten: Wenn Sie
        etwas hineingeworfen haben, ist es weg!
\end{itemize}

Bei Änderungen der Reihenfolge von Abbildungen oder Tabellen, was während des Review-Prozess häufig vorkommt, müssen diese entsprechend neu durchnummeriert werden. Die dafür notwendigen Umbenennungen werden mittels eines Python-Programms (abgelegt auf dem IWT-Fileserver unter $\rightarrow$Austausch $\rightarrow$Forschungsdatenmanagement $\rightarrow$DataManagement\-Guidelines-ServiceFiles  $\rightarrow$ImageSorter) sowohl für in Word als auch in LaTeX geschriebene Manuskripte erleichtert.


\paragraph{Poster und Präsentationen}

\begin{table}[!h]
  \caption{%
  Die Daten jedes Beitrags müssen in sieben Unterverzeichnissen gespeichert
  werden; weitere Hinweise: \\
  ** Wenn das Poster präsentiert wurde, lädt der jeweilige Präsentator die
  letzte und überarbeitete Version (d.h. nur die wirklich verwendeten, aber
  absolut vollständigen Daten) in das geschützte Verzeichnis \\
  ***Wenn die in einer Abbildung/Tabelle wiedergegebenen Daten aus verteilten
  Rohdatenverzeichnissen stammt, genügt es, hier nur das erzeugende Programm zu
  speichern%
  }
\begin{tabularx}{\linewidth}{l|p{1mm}X}
  \toprule
  \midrule
  \multirow{2}{*}[-17pt]{\texttt{00\_FinalPoster}$^*$} &
    \,\tabitem & Die allerletzte Version Ihres Posters als pdf mit Band, Jahr
                 und Seitenzahlen auf dem VT-Server in 'VT-Publikationen' \\
    & \,\tabitem & EndNote-Eintrag für IWT-WiKo mit DOI-Nummer, und ein EndNote
                   *.ris ASCII-Export, der den FAIR-Prinzipien entsprichts \\
  \midrule
  \texttt{01\_Poster+Abstract} &
    \,\tabitem & das *.pdf und *.pptx or *.tex \\
  \midrule
  \multirow{1}{*}[-25pt]{\texttt{02\_Figures}$^{**}$} &
    \,\tabitem & ein Unterverzeichnis für jede Abbildung (figure\_01, ...) mit
                 Datenpunkten als*.csv or *.txt, oder die erzeugende Datei/ das
                 erzeugende Programm für die Abbildungspunkte (Matlab, Python,
                 Origin,...); die Ergebnisse stammen aus Daten, die im
                 Verzeichnis \texttt{04\_PrimaryData} gespeichert oder daraus
                 extrahiert wurden \\
  \midrule
  \multirow{1}{*}[-7pt]{\texttt{03\_Tables}$^{**}$} &
    \,\tabitem & ein Unterverzeichnis für jede Tabelle (Tabelle\_01, ...) in dem
                 die Daten in Form einer csv-Datei gespeichert  \\
  \midrule
  \multirow{2}{*}[-8pt]{\texttt{04\_PrimaryData}} &
    \,\tabitem & gemessene oder simulierte Daten, jedes Unterverzeichnis mit
                 einer "readme.json" Datei zur Beschreibung der Daten \\
    & \,\tabitem & ELN pdf mit QR-code (siehe unten) \\
  \midrule
  \multirow{3}{*}{\texttt{05\_ProgramSources}} &
    \,\tabitem & e.g. OpenFOAM-Solver, Python/Matlab code, etc. \\
    & \,\tabitem & Fallbeispiele (Netze) for OpenFOAM, Fluent, ABAQUS etc. \\
    & \,\tabitem & “readme.json” Datei mit Metadaten \\
  \midrule
  \multirow{2}{*}{\texttt{06\_References}} &
    \,\tabitem & die Referenzen *.bib oder*.enl zusammen mit *.ris \\
    & \,\tabitem & Zitierte Arbeiten/Bücher (wenn Veröffentlichungsrechte
                   erlaubt sind) \\
  \midrule
  \multirow{1}{*}{\texttt{07\_Supplement}} &
    \,\tabitem & Ergänzende Dateien \\
  \midrule
  \multirow{1}{*}{\texttt{08\_Misc}} &
    \,\tabitem & Konferenzprogramm  \\
  \midrule
  \bottomrule
\end{tabularx}

\label{table:poster-directory-structure}
\end{table}
\noindent Poster oder Präsentationen werden z.B. gespeichert unter: \\
\texttt{/VT-Server/<Abteilung>/Projekte/<Förderung>/<Projektname>/Poster/} \\
\texttt{/VT-Server/<Abteilung>/Projekte/<Förderung>/<Projektname>/Präsentationen/} \\
%
Die Verzeichnisstruktur ist ähnlich wie bei den Papers (siehe Tabelle \ref{table:poster-directory-structure}), d.h. Sie müssen hier
wieder alle verwendeten Mess- oder Simulationsdaten speichern. Einige
Verzeichnisse haben abweichende Namen und einen anderen Inhalt, wie z.B.
\texttt{"01\_Poster+Abstrakt"}, das Poster und Abstrakt der Konferenz enthält.
Falls Sie eine mündliche Präsentation gehalten haben, heißen die ersten beiden
Verzeichnisse \texttt{"00\_FinalPresentation"} und
\texttt{"01\_Praesentation+Abstrakt"}.


Bei Änderungen der Reihenfolge von Abbildungen oder Tabellen müssen diese entsprechend neu durchnummeriert werden.
