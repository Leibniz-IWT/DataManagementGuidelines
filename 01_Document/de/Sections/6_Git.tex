\section{Git}

Git ist ein webbasiertes Tool, das ein Repository für die Versionskontrolle
bereitstellt. Es wird verwendet, um Ihre Softwareentwicklungen zu verwalten, zu
planen, zu erstellen, zu verifizieren und zu überwachen. Außerdem kann es als
System zur Versionierung von Dokumenten verwendet werden. Bei der Erstellung
eines größeren Textes (Dissertation, Referate, Anträge, ...) kann zum
Beispiel der aktuellen Stand des Dokuments eingeben werden, und alle vorherigen
Versionen werden mitgespeichert. Damit lässt sich jeder geänderte Satz
nachvollziehen und rekonstruieren.\\
Wir haben derzeit zwei Git-Instanzen am IWT:
\begin{itemize}
  \item Ein GitLab als Teil des VT-Servers für nicht-öffentliche Dokumente
        \begin{itemize}
          \item[$\rightarrow$] Anmeldung bei den Administratoren siehe in \nameref{ssc:Dateiorganisation} \ref{ssc:Dateiorganisation}
        \end{itemize}
  \item Öffentliches Git: \url{https://github.com/Leibniz-IWT/}
        \begin{itemize}
          \item[$\rightarrow$] Anmeldung: Email mit dem Betreff “IWT Organization access
                               request” an: 
                               \href{mailto:github@iwt.uni-bremen.de}%
                               {github@iwt.uni-bremen.de}
        \end{itemize}
\end{itemize}
