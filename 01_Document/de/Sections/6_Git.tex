\section{Git Software Repositorium}

Git ist ein webbasiertes Tool, das ein Repository für die Versionskontrolle
bereitstellt. Es wird verwendet, um Ihre Softwareentwicklungen zu verwalten, zu
planen, zu erstellen, zu verifizieren und zu überwachen. Außerdem kann es als
System zur Versionierung von Dokumenten verwendet werden. Bei der Erstellung
eines größeren Textes (Dissertation, Referate, Anträge, ...) kann zum
Beispiel der aktuellen Stand des Dokuments eingeben werden, und alle vorherigen
Versionen werden mitgespeichert. Damit lässt sich jeder geänderte Satz
nachvollziehen und rekonstruieren.

Die Git-Instanz am IWT ist hier zugänglich: \url{https://github.com/Leibniz-IWT/}.
Anmeldung mittels Email und Betreff “IWT Organization access request” an:
\href{mailto:github@iwt.uni-bremen.de}%
{github@iwt.uni-bremen.de}
