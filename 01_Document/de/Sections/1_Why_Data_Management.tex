\section{Warum Datenmanagement?}

\textbf{Zeitersparnis} \\
Wenn Sie Ihre Datenverwaltung planen, sparen Sie Zeit und Ressourcen, z.B.
durch schnelleres Auffinden von Daten. Gut verwaltete Daten erfordern weniger
Vorbereitungen für die Weitergabe oder das Verfassen eines Artikels. \\[6pt]
%
\textbf{Aufbewahrung Ihrer Daten} \\
Wenn Sie Ihre Daten auf einem Dateiserver (Repository) ablegen, sichern Sie
die von Ihnen geleistete Arbeit und bewahren Ihren Forschungsbeitrag für
sich und andere auf. \\[6pt]
%
\textbf{Erhöhte Sichtbarkeit Ihrer Arbeit/Forschung} \\
Daten anderen Forschenden zur Verfügung stellen erhöht die Sichtbarkeit und damit die Relevanz Ihrer Forschung. \\[6pt]
%
\textbf{Datenintegrität erhalten} \\
Die Verwaltung und Dokumentation Ihrer Daten während deren Werdegangs
ermöglicht es Ihnen und anderen, Ihre Daten in der Zukunft zu verstehen und
weiter zu verwenden.\\[6pt]
%
\textbf{Einhaltung der Anforderungen von Mittelgebern} \\
Viele Mittelgeber (z.B. DFG) verlangen, dass die im Rahmen eines
Forschungsprojekts gesammelten Daten archiviert werden.\\[6pt]
%
\textbf{Förderung neuer Erkenntnisse} \\
Die gemeinsame Nutzung Ihrer Daten mit anderen Forschenden kann durch
Methoden des maschinellen Lernens zu neuen und unerwarteten Entdeckungen führen
und Forschungsmaterial für diejenigen bereitstellen, die über wenig oder keine
finanziellen Mittel verfügen. \\[6pt]
%
\textbf{Unterstützen Sie den öffentlichen Zugang zu Daten} \\
Seien Sie Impulsgeber für die Forschung. Zeigen Sie Ihre Unterstützung für die
offene Wissenschaft, indem Sie Ihre Daten zur Verfügung stellen.
