\section{FAIR Data und Open Science}\label{ssc:FAIR-Data}

FAIR (Findable, Accessible, Interoperable, Reusable) werden Daten dann genannt, wenn sie im Internet gefunden, herunterladbar und tatsächlich auch genutzt werden können. Insbesondere für das in Kapitel \ref{ssc:data-for-publication} geforderte Verzeichnisstruktur der Datenablage gilt die Anforderung, Daten FAIR abzulegen, d.h. Messdaten müssen klar nachvollziehbar sein (Dokumentation des verwendeten Messgerät mit allen Einstellungen, Resultate mit Einheiten als ASCII-File, etc.) und Simulationsdaten müssen nachgerechnet werden können (Quellcode, Rechengitter, Randbedingungen etc.; keine kommerziellen Programmpakete, aber case-Dateien schon). Mit diesem Vorgehen wird die Gute Wissenschaftliche Praxis (GWP) realisiert.

Für mit öffentlichen Geldern geförderten Projekte (und das sind die meisten am L-IWT) gilt zusätzlich der Grundsatz, dass alle im Projekt gewonnen Daten und entwickelten Methoden der Öffentlichkeit zugänglich gemacht werden müssen. Dazu kann die oben beschriebene Verzeichnisstruktur auf zenodo \url{https://zenodo.org/}, einem Repositorium für Wissenschaftliche Daten, hochgeladen werden, wofür dann eine DOI bereitgestellt wird. Allerdings dürfen die PDFs bei Veröffentlichung in Fachzeitschriften nur wenn dies explizit erlaubt ist (was genau geprüft werden muss) mit abgelegt werden, oder nach einer gewissen Embargo-Zeit. In der Regel darf aber eine selbst formatierte Version des Textes samt Grafiken veröffentlicht werden.

Daten aus anderen Forschungseinrichtungen, die z.B. in einer Kooperation mit in eine Veröffentlichung einfliessen, müssen entsprechend unserer Vorgaben ebenfalls FAIR abgelegt werden, also entweder uns überlassen und dann nach Kapitel \ref{ssc:data-for-publication} abgespeichert, oder mittels Verlinkung auf ein Repositorium zugänglich gemacht werden.

Damit die abgelegten Daten im Sinne von FAIR von anderen Personen weitergenutzt werden können, müssen sie mit entsprechenden Lizenzen versehen werden (ohne Angabe ist keine Weiternutzung und nur das Betrachten erlaubt).
%siehe https://academia.stackexchange.com/questions/63139/public-dataset-without-license-what-is-allowed
Dazu hat es sich bewährt, Texte unter der 'CC-BY-Lizenz' und Software unter der '3-clause BSD license' (oder 'BSD-3') abzulegen.

Im Rahmen von Open Science ist FAIR ein Teil von Open Data, das zusammen mit Open Access von Fachartikeln und Open Source von Software drei wesentliche Prinzipien für eine offene Wissenschaft bilden. Weitere Informationen, u.a. mit dem Leibniz Open Science Leitbild, sind auf der IWT Webseite zu finden\footnote{https://www.iwt-bremen.de/en/institute/about-us/open-science}.
