\section[Electronic Lab Notebooks]{Electronic Lab Notebooks (ELNs)}\label{ssc:ELN}

Electronic Lab Notebooks (ELNs) ermöglichen es Forschende, experimentelle
Verfahren, Protokolle, Notizen und Daten über ihren Computer oder ein mobiles
Gerät zu organisieren und zu speichern. ELNs bieten gegenüber dem traditionellen
Papiernotizbuch mehrere Vorteile bei der Dokumentation von Forschungsarbeiten in
der aktiven Phase eines Projekts. Dazu gehören die Durchsuchbarkeit innerhalb
und zwischen den Notizbüchern, die sichere Speicherung mit mehreren Redundanzen,
der Fernzugriff auf die Notizbücher und die Möglichkeit, die Beschreibungen von
Experimenten oder Simulationen problemlos mit anderen Teammitgliedern und
Mitarbeitenden zu teilen.

\subsection{eLabFTW}

Dies ist ein OpenSource, generisches, browserbasiertes ELN,
entwickelt/initialisiert von Nicolas Carpi (Institut Pasteuer, Paris) im Jahr 2012. Die
Daten werden in MySQL/MariaDB gespeichert. Es wird von vielen Entwickler*innen
gepflegt und ist weltweit im Einsatz (Berkley, Indian Institute of Technology,
KIT, ...).
\begin{itemize}
  \item Zugang: \url{https://eln.mvt-bremen.de/} oder \url{https://134.102.38.139}
  \item Frei definierbarer Status der Experimente ("beendet", "läuft", ...)
  \item Definition von Vorlagen für eine schrittweise Ablaufbeschreibung von
        Messungen
  \item Jedes Experiment erhält eine eindeutige ID und kann leicht als pdf-Datei
        zusammengefasst werden
  \item Frei definierbare Kategorien/Tags
  \item Grafischer Texteditor zur Beschreibung Ihrer Experimente und Simulationen
  \item Dateianhänge von Daten mit Vorschau der gängigen Formate (pdf, tiff,
        png, ...)
  \item Verknüpfung von Experimenten
  \item Zeitstempel-Dienst (RFC 3161, z.B. DFN)
  \item Datenimport/-export (csv, zip, json, ...)
  \item Zugriff mit z.B. Python über eine Anwendungsprogrammierschnittstelle (API)
\end{itemize}

Sie müssen für jedes Experiment eine PDF-Datei erstellen, indem Sie im Bearbeitungsmodus einfach auf 'PDF erstellen' klicken (siehe
\texttt{‘04\_PrimaryData’} in Tabelle \ref{table:paper-directory-structure}).
Diese Art der Dokumentation enthält einen eindeutigen QR-Code und ist für Daten,
die veröffentlicht werden, erforderlich! Der QR-Code ermöglicht einen direkten
Link zu den Daten zusammen mit einer eindeutigen ID.

Die \textbf{maximale Größe} der hochgeladenen Daten sollte \textbf{100Mbyte pro
Datei} nicht überschreiten. Weitere Tipps und Tricks finden Sie im
seafile-Dokument (bitte fügen Sie dort Kommentare hinzu, wenn Sie neue
Möglichkeiten der Nutzung entdecken): \\
\url{https://seafile.zfn.uni-bremen.de/f/fe685882eab14a2e853c/}
