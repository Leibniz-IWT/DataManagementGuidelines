\thispagestyle{empty}
\begin{center}
  \large\headerfont\fontseries{m}\textcolor{iwtdark}{Vorwort}
\end{center}

\begin{quote}
  "`Am 20. Juli 1969 kletterte Neil Armstrong aus seinem Raumschiff und setzte
  seine Füße auf den Mond. Die Landung wurde live in die ganze Welt übertragen
  und war ein bedeutendes Ereignis in der Geschichte der Wissenschaft und der
  Menschheit. Heute können wir uns immer noch das grobkörnige Video der
  Mondlandung ansehen, aber wir können uns nicht das Originalmaterial in höherer
  Qualität ansehen oder einige der Daten dieser Mission untersuchen. Das
  liegt daran, dass viele der Daten aus der frühen Weltraumforschung für immer
  verloren sind. ($\ldots$) Die Bänder wurden wahrscheinlich irgendwann in den 1970er
  Jahren gelöscht und für die Datenspeicherung wiederverwendet". \\
  \null\hfill - \citeauthor{briney2015}\cite{briney2015}
\end{quote}

\noindent Dieser Leitfaden (https://github.com/Leibniz-IWT/DataManagementGuidelines) gilt für alle Mitarbeitenden des Leibniz-Instituts für Werkstofforientierte Technologien (L-IWT) und
dient als Nachschlagehilfe für die Datensicherung. Es geht dabei um die Frage,
warum wir über Datenmanagement nachdenken sollten, und enthält Anleitungen und
Beispiele. Datenmanagement ist im Informationszeitalter eine unabdingbare
Voraussetzung und Grundlage dafür, die Fülle an Daten zu bewältigen. Es hilft,
die eigenen Daten zu organisieren, ist aber auch für die Weitergabe von Daten
unerlässlich, damit andere die Struktur verstehen und die Informationen aus
Ihren Daten korrekt deuten können. Auf diese Weise entsprechen Ihre Daten dem
\textbf{FAIR}-Standard (Findable, Accessible, Interoperable and Reusable), der
Grundlage für die neue Disziplin der \textbf{Datenwissenschaft}.

Es gibt viele Themen im Bereich des Datenmanagements, wie z.B. Data Governance
und Sicherheit oder Speichermanagement, aber wir behandeln hier nur einen
komprimierten Auszug, der auf den Umgang mit Daten in unserem Institut
zugeschnitten ist. Dieser Reader soll Ihnen helfen zu klären, was derzeit unter dem Begriff \textbf{Data Steward} zu verstehen ist. Er lehnt sich an die von der Leibniz-Gemeinschaft heraus gegebenen Leitlinien an \cite{leibniz2018}.
