\section{Pläne zur Datenverwaltung}

Der Umgang mit den Daten im Rahmen eines Projekts beinhaltet verschiedene
Aspekte, die in einem Datenmanagementplan (DMP) beschrieben werden. Ein DMP
enthält strukturierte Informationen über den Forschungsprozess des jeweiligen
Projekts. Diese Informationen müssen jedoch vor Projektbeginn vorliegen:
Antragsteller müssen in jedem Antrag einen DMP erstellen, der von allen
wichtigen Forschungsförderorganisationen (DFG, BMBF, etc.) gefordert wird.
Die Antragsteller machen sich also schon beim Schreiben des Antrags Gedanken
darüber, wie sie mit den Daten umgehen wollen. Die folgenden Themen gehören zu
den typischen DMPs und könnten für Sie von Interesse sein; weitere Details
finden Sie in \cite{dfg2021,hannover2020}:
\begin{enumerate}
  \item Administrative Informationen: Projektname, Art der Finanzierung, Zeitraum.
  \item Methoden der Datenerzeugung: Simulationen, Experimente (Geräte),
        Datenumfang, Art der Datendokumentation.
  \item Datensicherheit: Speicherort, Speicherintervall und -kapazität,
        wer hat Zugriff.
  \item Archivierung: welche Daten werden wo mit welchen Metadaten versehen.
  \item Gemeinsame Nutzung von Daten: welches Repository, Lizenzbedingung,
        welche erforderlichen Metadaten
  \item Ressourcen und Zuständigkeit: wer ist zuständig für Prozesse, IT,
        Festlegung von Vorgaben und Formaten, Überwachung; erforderliche
        personelle Ressourcen; Kosten.
\end{enumerate}
Weitere Einzelheiten finden Sie im Anhang; einige beispielhafte DMP sind auch hier
zu finden: \\
\url{https://www.cms.hu-berlin.de/de/dl/dataman/arbeiten/dmp_erstellen}
