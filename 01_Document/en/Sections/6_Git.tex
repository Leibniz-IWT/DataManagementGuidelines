\section{Git}

Git is a web-based tool that provides a repository for the version control.
It is used to manage, plan, create, verify and monitor your software
developments. Furthermore, it can be used as a document versioning system.
For example, during the creation of a larger piece of writing (a dissertation,
papers, proposals, ...), you can enter the actual state of the document,
and all the previous versions are stored as well. With that, every changed
sentence can be tracked and reconstructed.

We have actually two Git instances on IWT:
\begin{itemize}
  \item GitLab as part of the VT-Server for non-public documents
        \begin{itemize}
          \item[$\rightarrow$] registration by the admins mentioned above
        \end{itemize}
  \item Public Git: \url{https://github.com/Leibniz-IWT/}
        \begin{itemize}
          \item[$\rightarrow$] registration: email `IWT Organization access
                               request' to:
                               \href{mailto:github@iwt.uni-bremen.de}%
                               {github@iwt.uni-bremen.de}
        \end{itemize}
\end{itemize}
