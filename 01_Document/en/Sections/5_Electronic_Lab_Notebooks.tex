\section[Electronic Lab Notebooks]{Electronic Lab Notebooks (ELNs)}\label{ssc:ELN}

Electronic Lab Notebooks (ELNs) enable researchers to organize and store
experimental procedures, protocols, notes and data using their computer or
mobile device. ELNs can offer several advantages over the traditional paper
notebook in documenting research during the active phase of a project, including
searchability within and across notebooks, secure storage with multiple
redundancies, remote access to notebooks, and the ability to easily share
notebooks among team members and collaborators.

\subsection{eLabFTW}

This is an OpenSource, generic, browser based ELN, developed/initialized by
Nicolas Carpi (Institut Pasteuer, Paris) 2012. Data are stored in MySQL/MariaDB.
It is maintained by many developers and is used worldwide (Berkley, Indian
Institute of Technology, KIT, ...).
\begin{itemize}
  \item Access: \url{https://eln.mvt-bremen.de/} or \url{https://134.102.38.139}
  \item Free definable status of the experiments (“finished”, “running”, ...)
  \item Definition of templates for a step-by-step procedure description of
        measurements
  \item Every experiment gets a unique ID and can be summarized easily as a pdf
  \item Free definable categories/tags
  \item Graphical text editor to describe your experiments and simulations
  \item File attachments of data with preview of common formats (pdf, tiff, png, …)
  \item linking of experiments
  \item Time stamp service (RFC 3161, e.g. DFN)
  \item Data import/export (csv, zip, json, …)
  \item access using, e.g., Python via an Application Programming Interface (API)
\end{itemize}

You have to create a pdf for every experiment simply by a click (‘Make a pdf’)
in the edit mode (see \texttt{‘04\_PrimaryData’} in Table
\ref{table:paper-directory-structure}. This kind of documentation includes a
unique QR-code and is required for data which is published! The QR-code enables
a direct link to the data together with unique ID.

The \textbf{maximum size} of uploaded data should not exceed
\textbf{100Mbyte per file}. Further hints and tricks may be found in the seafile
document (please add comments there if you explore new ways of usage): \\
\url{https://seafile.zfn.uni-bremen.de/f/fe685882eab14a2e853c/}
