\section{FAIR Data}

FAIR (Findable, Accessible, Interoperable, Reusable) is a label for data which are findable in the Web, downloadable and can be used for own purposes. The data in the directory structure described in chapter \ref{ssc:data-for-publication} have to be stored in a FAIR way. This means that data from measurements must be clearly understandable (description of the used device with all adjustments, results with units as ASCII files, etc.) and data from simulations must be reproducible (source code, mesh, boundary conditions, etc.; no commercial programs, but their case file). This method also guarantees the fulfilment of Good Scientific Practice.

For all projects funded by public sponsors (and these are the prevailing case on L-IWT) there hold the principle that all gained data and developed methods have to be provided to the public. Therefore, the directory structure described above can be uploaded to zenodo \url{https://zenodo.org/}, a repository for scientific data, where you get a DOI for your data. But note: PDFs from a publisher can only be stored there if this is permitted explicitly by the journal and must be checked carefully. Usually, it is allowed to provide text and diagrams in an document formated by your own.

Data from other research institutes which are included in a publication, e.g. due to a cooperation, have to be saved FAIR according to our guidelines. This means they provide us their data (stored according to chapter  \ref{ssc:data-for-publication}). If they refuse this, they have to provide at least their data on a public repository.

The reuse of your data by others in the sense of the FAIR principles requires the link to a licence (without a statement the default term is "all rights reserved" which only allows viewing but no further usage).
%see https://academia.stackexchange.com/questions/63139/public-dataset-without-license-what-is-allowed
Therefore, common licences are the 'CC-BY' licence for text and  the '3-clause BSD' license (or 'BSD-3') for software.

The FAIR principles belong to Open Data and is one part within the framework of Open Science, with Open Access of scientific papers and Open Source of software as the three main components.
