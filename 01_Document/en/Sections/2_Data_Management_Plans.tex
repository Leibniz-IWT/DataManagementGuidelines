\section{Data Management Plans}

The handling of the data, created during a project, contains several issues
which are described in a Data Management Plan (DMP). A DMP contains structured
information about the research process of the corresponding project. But these
information has to be given long before a project is started: Applicants have
to create a DMP in every proposal, required by all important research funding
organizations (DFG, BMBF, etc.). So in the process of writing, applicants start
to think about how they will treat data. The following topics belongs to typical
DMPs and might be of interest to you; more details can be found in
\cite{dfg2021,hannover2020}:
\begin{enumerate}
  \item Administrative information: project name, kind of funding, time period.
  \item Methods of data generation: simulations, experiments (devices),
        data size, kind of data documentation.
  \item Data safety: location, interval and capacity of storage device,
        who gets access.
  \item Archiving: which data are archived where with what kinds of metadata.
  \item Data sharing: which repository, license condition, required metadata.
  \item Resources and responsibility: who is responsible for processes, IT,
        defining defaults and formats, monitoring; required personal resources;
        costs.
\end{enumerate}
Please see the Appendix for more details; some exemplary DMPs can be found here:
\url{https://www.cms.hu-berlin.de/de/dl/dataman/arbeiten/dmp_erstellen}
