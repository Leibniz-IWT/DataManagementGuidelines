\section{File Organization}
The following explains what and how to save your files on our servers:
\begin{itemize}
  \item Access to the VT-Server is established with your Web browser by typing
        \textbf{\url{http://134.102.40.154}} or
        \textbf{\url{http://vt-nas.mvt-bremen.de}}
        in the address field, via mapping the VT-Server to the Windows Explorer
        as an own drive, or by using the Synology Drive Client
        (see \cite{synology2022}) with some more and very helpful options like,
        e.g., document versioning. The \textbf{maximum size} of uploaded data
        should not exceed \textbf{100Gbyte per file}. Access privilege is
        administrated by
        Stefan Endres (\href{mailto:s.endres@iwt.uni-bremen.de}%
                       {s.endres@iwt.uni-bremen.de}),
        Arvind Chouhan (\href{mailto:a.chouhan@iwt.uni-bremen.de}%
                        {a.chouhan@iwt.uni-bremen.de})
        and Nils Ellendt (\href{mailto:ellendt@iwt.uni-bremen.de}%
                          {ellendt@iwt.uni-bremen.de}).
  \item The server of the manufacturing technology (FT) division is accessed via
        \textbf{\textbackslash\textbackslash fts-ags.iwt.uni-bremen.de}
        or \textbf{\textbackslash\textbackslash fts-ags}
        or \textbf{\textbackslash\textbackslash 134.102.244.1} in the
        windows explorer. Access privilege is administrated by the IWT IT
        division.
\end{itemize}

\subsection{Different Data Types}
Depending on the origin of the data, we distinguish between two data types:
\begin{itemize}
  \item Primary data (sometimes called raw data) is generated by measurement
        devices, sensors/cameras, simulation programs, observations etc. and
        are untreated; it is collected for the first time by the researcher.
  \item Secondary (or analyzed) data are processed primary data. (In social
        science, secondary data are already collected or produced data by
        others).
\end{itemize}
Very large primary data cannot be saved on our ELN%
\footnote{The use of the Electronic Lab Notebook (ELN) is currently only
mandatory for the VT. },
but the origin of that data are discussed there together with information about
the storage location (e.g. somewhere on a file server). The decision about the
limit, which data is considered as large, is:
\begin{itemize}
  \item data < 100 MByte is stored and documented on the ELN
  \item 100 MByte  < data < 1TByte is stored on the VT-Server and documented
        on the ELN
  \item data > 1 TByte is stored on USB-Disks and documented on the ELN
\end{itemize}
It is absolutely required that \textbf{all data from external sources} (e.g. by an USB
stick) have to be checked for viruses manually, if the auto scan doesn’t come
up, before they are used/stored on hard disk.

\subsection{File and Folder Naming}

Spend time planning out file naming conventions in the beginning of a project.
How do you or others will look for and access files at a later date? Do you
think about them by type, location, study or something else? \\[8pt]
Naming conventions:
\begin{itemize}
  \item File names are self-explanatory - they tell what data is in!
  \item Do not use spaces or special characters - use CamelCase and hyphen
        ‘-’ or underscore ‘\_’ as separator: \\
        \begin{tabular}{ll}
          \textbf{NO}  & ‘name date v1.txt’ \\
          \textbf{YES} & ‘name\_date\_v01.txt’ \\
          \textbf{YES} & ‘nameDateV01.txt’ \\
          \textbf{YES} & ‘simpleFoam\_u0.2m-s\_etha2mPas\_v01’ \\
        \end{tabular}
  \item Develop a file naming scheme that includes information about the data.
        Example: \\
        {[Date]}\_{[Run]}\_{[SampleType]}
  \item Consider sorting when deciding what element of the file name will
        go first: \\
        Use ‘YYYY-MM-DD’ to save dates. \\
        Use leading zeros: \\
        \begin{tabular}{lll}
          \textbf{NO}  & ‘ProjID\_v1.csv’  	...  	‘ProjID\_v11.csv’ & \\
          \textbf{YES} & ‘ProjID\_v01.csv’ 	... 	‘ProjID\_v11.csv’
             & (sequence up to 99) \\
          \textbf{YES} & ‘ProjID\_v001.csv’ 	...	‘ProjID\_v111.csv’
             & (sequence up to 999) \\
        \end{tabular}
\end{itemize}

\subsection{File Formats}
As technologies changes, researchers should plan for both hardware and software
obsolescence and consider the longevity of their file format choices to ensure
long term readability and access.


File formats more likely to be accessible in the future have the following
characteristics:
\begin{itemize}
  \item Non-proprietary
  \item Open, documented standard
  \item Common usage by research community
  \item Standard representation (ASCII, Unicode)
  \item Unencrypted
  \item Uncompressed (if not too large...)
\end{itemize}

\noindent Examples of \textbf{preferred} file format choices include:
\begin{itemize}
  \item Open Document Files: .odt (Text), .ods (Spreadsheet),
        .odp (Presentation); primary format of OpenOffice/Libre Office,
        standardized in ISO 26300
  \item Office Open XML: .docx (Text), .xlsx (Spreadsheet),
        .pptx (Presentation); primary format of Microsoft Office,
        standardized in ISO 29500
  \item Ritch Text Format: .rtf; proprietary, but well documented format
  \item Plain Text or Markup Language: .txt, .md (Markdown), .html (Hypertext),
       .tex (LaTeX), .csv (comma separated values)
  \item Portable Document Format: .pdf; standardized in ISO 32000
  \item Bibliographic Data: .enl (EndNote; this is the standard software for
        literature management at the IWT); otherwise, please use human readable
        formats such as .bib (BibTex), .ris (Research Information System Format)
  \item Video files in data formats .mp4 and .mpg generated by codecs H.264 and
        H.265; standardized format (ISO 14496-10) for video compression;
        \textbf{DO NOT USE proprietary codecs such as Quicktime (.mov, .qt)}
  \item mages: .jpg is the standard file format of most cameras, it is readable
        by almost any image-related tools, but is not ideal in all cases (e.g.
        line art). It offers no transparency. Moreover, it is a lossy format
        which can lead to a permanent degradation of the image, according to the
        chosen compression. Thus it should be avoided if this is not tolerable.
        The succession format JPEG2000 .jp2 allows lossless compression and thus
        may be used for image archiving. Due to licensing issues it is not
        widely used. The .tiff (Tagged Image File Format) is an important format
        for file exchange with publishers. It allows lossless compression and
        high depth (32 bit) as well as CMYK color. It may be used for image
        archiving, however, the file size remains comparably large. The
        "Portable Network Graphic" format .png is a raster image format with
        good lossless compression. It yields larger file sizes than .jpg for
        photos but is very well suited for line art.
  \item Drawings: It is preferable to use vector based file formats for sketches
        and drawings, as these allow for an unlimited scaling of the image
        without affecting the quality. For large printouts, vector graphics are
        significantly smaller than raster graphics. However, vector-based images
        are not handled well by all word processors and even if they are usable,
        may yield varying results on different computers. It is recommended to
        prepare a vector graphic in an open format (.svg, scalable vector
        graphics) to ensure long-term readability and it export it to the file
        format best suited for your word processing software (e.g. .emf for
        Word, .eps or .pdf for LaTeX).
  \item Binary data: sometimes, it is most efficient to save data in a binary
        format, as text-based representations would dramatically increase the
        file size. In this case, it is vital to document how the data is stored
        and preferably give an example of how it can be read again.
\end{itemize}

\noindent If you consider exporting your (measurement) data into a format
with the above characteristics, keep a copy in the original software format
unless you can assure that all data and metadata is correctly converted. If you
deposit your data in a repository, your files may be migrated to newer formats,
so that they’re usable to future researchers.

\subsection{Folder Hierarchy}

The specific value of data has a wide variance. Many measurements like those
from preliminary experiments were very important to design your experiments,
but they are more or less worthless in respect of publications. In contrast,
some of your data are highly valuable and find their way into a publication.
Due to these differences, the location where you save your data has different
valence, according to these three data kinds:
\begin{enumerate}[label=\Roman*.]
  \item Data with a low valence, see:
       \ref{ssc:daily-data} \nameref{ssc:daily-data}
  \item Data which are going to be published, see:
        \ref{ssc:data-for-publication} \nameref{ssc:data-for-publication}
  \item Data of accepted publications: These data are stored on a special area
        within your fileserver where you can only write your data one time – you
        cannot change or delete anything there. It is like a mail box: Once you
        have thrown in your tax declaration, it is over. This data area is for
        accepted publications only where really everything is all right!
\end{enumerate}

\subsubsection{Daily Data}\label{ssc:daily-data}

Every day you generate data, for instance by measuring, simulations, preparing a
paper etc. The VT-Server serves as a file storage for your data, and all these
daily data are saved under \texttt{Mitarbeiter} and your
\texttt{<family name>}: \\
\texttt{/VT-Server/<department>/Mitarbeiter/<family name>/} \\
The names enclosed by \texttt{<department>} have to be chosen from our
departments:
\begin{itemize}
  \item MPS (MehrPhasenStrömung)
  \item RST (Reaktive Sprüh-Technik)
  \item SPK (SPrühKompaktieren)
\end{itemize}
according to your affiliation. An example directory is: \\
\texttt{/VT-Server/MPS/Mitarbeiter/riefler/} \\
\\
Here, you save the following data:
\begin{itemize}
  \item[$\rightarrow$] \textbf{For Experimental Data:}
    \begin{itemize}
      \item All experiments of the project
      \item Relation between these experiments (which parameters are varied) in
            a JSON based text file, see \autoref{sc:data-documentation}
      \item intermediate results
      \item List of all used devices with adjustment parameters
    \end{itemize}
  \item[$\rightarrow$] \textbf{Experimental Setups:}
    \begin{itemize}
      \item Drawings of all parts
      \item A list of all required parts and devices
    \end{itemize}
  \item[$\rightarrow$] \textbf{For Simulation Data:}
    \begin{itemize}
      \item The required initial values and mesh information to recalculate all
            simulations of the project
      \item Relation between these simulations (which parameters are varied) in
            a JSON based text file, see \autoref{sc:data-documentation}
      \item Intermediate results
      \item List of the used software together with used source code
      \item All required input files to reproduce simulations
    \end{itemize}
\end{itemize}

\noindent There is no prespecified directory structure for your daily data, so
that’s up to you. However, these data under your name are distinct from project
data like, e.g., important experimental findings or simulations which are
published, which are described in the following.

\subsubsection{Data for Publication}\label{ssc:data-for-publication}

Here, you save all data relevant for every publication within this project.
Depending on the size of the data, and the decision within your department (see
above), primary data files larger than, say, 10 GByte must not be saved within
the directory specified here. Instead, a reference has to be given within the
documentation in the experiment description on the ELN about the corresponding
origin (e.g. high-speed camera measurement or CFD simulation) and where exactly
the primary data are stored. The decisive \textbf{basic principle} for all your
raw and primary data is:
\textbf{\underline{All data must be understandably documented in its origin!}}

When you prepare a document, say a presentation or a project report, the data
are saved (according to a strict scheme given later) under this directory: \\
\texttt{/VT-Server/<department>/Projekte/<funding>/<project name>}
\begin{itemize}
  \item[$\rightarrow$] \textbf{Resulting Documents:}
    \begin{itemize}
      \item Documents from your scientific work: Papers, Dissertations,
            Presentations, Reports, Proposals, etc.
      \item All results (figures, tables, movies, animations, ...) and the
            underlying measured or simulated data together with evaluation
            programs
    \end{itemize}
\end{itemize}
Funding sources (\texttt{<funding>}) are for instance: DFG, AiF, BMBF,
Industrie, ERC, etc. Last, the directory name for published works in a project,
the \texttt{<project name>}, is compound on: \\
\texttt{<year of project start>\_<department>\_<project handle>\_<family name>}
See \autoref{app:dummy-paper} for an example, and the two examples given here:
\begin{itemize}
  \item \texttt{/VT-Server/MPS/Projekte/DFG/2016-MPS-Tropfengenerator\_Riefler/...}
  \item \texttt{/VT-Server/RST/Projekte/DFG/2018\_RST\_Flatbandpotential\_Naatz/...}
\end{itemize}
The doubling of 'MPS' and 'RST' comes due to an anticipated common file server
which hosts data of all IWT departments.

\paragraph{Papers and Dissertations}

\noindent In agreement with all IWT departments, the directory naming scheme for
published data of the File-Server is based on departments, the funding source
and then the corresponding project name (see above for \texttt{<project name>}
definition):
\\
Text and data of a paper are saved, e.g., in that directory:  \\
\texttt{/VT-Server/<department>/Projekte/<funding>/<project name>/Papers/} \\
with a directory name like that:
\\
\texttt{<year>\_<department>\_<paper-name>\_<first author>} \\
Text and data of a dissertation are saved, e.g., in that directory:
\\
\texttt{/VT-Server/<department>/Projekte/<funding>/<project name>/Dissertation/} \\
with a directory name like that:
\\
\texttt{<year>\_<department>\_<dissertation\_title>\_<author>} \\
In these directories, you save all the data of your \textbf{paper} in
subdirectories, whose names are given in the left row of table
\ref{table:paper-directory-structure}, with the corresponding
files explained on the right.
\begin{table}[!h]
  \caption{%
  The data of every paper has to be saved in eight subdirectories;
  further remarks: \\
  **When the paper is ready for press the corresponding author loads the last and
  revised version (i.e. only the real used but absolutely complete data) into
  the protected directory \\
  ***If the data reproduced in a figure/table is from distributed primary data
  directories, it is sufficient to save here only the generating program%
  }
\begin{tabularx}{\linewidth}{l|p{1mm}X}
  \toprule
  \midrule
  \multirow{2}{*}[-17pt]{\texttt{00\_FinalPublication}$^*$} &
    \,\tabitem & The very last version of your paper as a pdf with volume, year
               and page numbers on the VT-Server into \texttt{’VT-Publikationen’} \\
    & \,\tabitem & EndNote entry for IWT-WiKo including DOI number, and an
                   EndNote *.ris ASCII export fulfilling the FAIR principles \\
  \midrule
  \texttt{01\_Manuscript} &
    \,\tabitem & the *.pdf and *.docx or *.tex \\
  \midrule
  \multirow{1}{*}[-25pt]{\texttt{02\_Figures}$^{**}$} &
    \,\tabitem & one sub-directory for every figure (\texttt{figure\_01}, ...)
                 with data points as *.csv or *.txt, or the generating
                 file/program for the figure points (Matlab, Python,
                 Origin,...); the results are from data stored in or extracted
                 from the \texttt{04\_PrimaryData} directory \\
  \midrule
  \multirow{1}{*}[-7pt]{\texttt{03\_Tables}$^{**}$} &
    \,\tabitem & one sub-directory for every table (\texttt{table\_01}, ...) with
                 data stored in as a csv-file \\
  \midrule
  \multirow{2}{*}[-8pt]{\texttt{04\_PrimaryData}} &
    \,\tabitem & measured or simulated data, each sub-directory with a
                 `readme.json' file describing the data \\
    & \,\tabitem & ELN pdf pages with QR-code (see below) \\
  \midrule
  \multirow{2}{*}{\texttt{05\_ProgramSources}} &
    \,\tabitem & e.g. OpenFOAM-Solver, Python/Matlab code, etc. \\
    & \,\tabitem & case files (mesh) for OpenFOAM, Fluent, ABAQUS etc. \\
  \midrule
  \multirow{2}{*}{\texttt{06\_References}} &
    \,\tabitem & the references *.bib or *.enl together with *.ris  \\
    & \,\tabitem & Cited papers/books (if publication rights are permitted) \\
  \midrule
  \multirow{1}{*}{\texttt{07\_Supplement}} &
    \,\tabitem & Supplementary files  \\
  \midrule
  \multirow{1}{*}{\texttt{08\_Misc}} &
    \,\tabitem & Reviewer comments / rebuttals  \\
  \midrule
  \bottomrule
\end{tabularx}

\label{table:paper-directory-structure}
\end{table}

In case of a \textbf{dissertation}, you save your presentation of the doctoral
examination together with the manuscript under the directory
\texttt{“01\_Manuscript+Presentation”}. If you do not have supplementary
information for your paper/dissertation, or you did not write specific computer
programs for your research, the \texttt{“05\_ProgramSources”},
\texttt{“07\_Supplement”} and \texttt{“08\_Misc”} directories must still be
there but with no entries.

An import directory is the \texttt{“02\_Figures”}. There, you save every figure
in the manuscript in good quality, and you also save a *.csv file with the data
in the figure, or you save a file which generates the corresponding figure –
for instance a Matlab file – from the primary data in the \texttt{“04\_RawData”}
directory. This data directory includes the measured or simulated data in *.txt
or *.csv format from the measuring device or the simulation software, and a
“readme.json” file generated by the Readme-File-Creator described below with
information and metadata about your measured / simulated data. The relation of
each data point in a figure to the primary data must be given either implicit,
for instance in the Matlab file, or better explicit in the “readme.json” file.

To complete your publication, every cited paper should be added to the directory
\texttt{“06\_References”} for internal use. All too often the ideas of a paper
cannot be tracked and reproduced because there are references which you can’t
get because your institution do not offer access to that specific journal, or
the reference are from a master thesis which is not available. Therefore, saving
all references with suitable file names for a direct relation – together with
all the previously mentioned data and information – presents your paper in its
entirety. Above that, EndNote allows a comfortable direct linking to the cited
publications, i.e. you should add these links in your ‘*.enl’ file. The same is
possible within BibTeX. However, if your paper will be transferred on a public
repository, only free publications can be uploaded because many publishing
companies, like Springer and Elsevier, do not allow to provide papers from their
journal or books.

Writing a paper is a multistep process. With the decision of the appropriate
journal, the process comes to a crucial point where all the data have to be
saved in the structure given above and in the following screenshot, and the
review process starts. If the paper is accepted, all changes are integrated
and the final pdf with volume, year and page numbers is released, then the last
actions are:
\begin{itemize}
  \item Check if every ELN-file which describes the experiment/simulation
        is saved.
  \item The very last step: Make a copy of the complete paper data into the
        directory \texttt{’Published’} on the file server. It’s a special
        directory: You can copy something there, but afterwards you cannot
        change anything. It’s like a post box: Once you have thrown something
        in, it is gone!
\end{itemize}

\paragraph{Posters and Presentations}

\noindent Posters or presentations are saved under: \\
\texttt{/VT-Server/<department>/Projekte/<funding>/<project name>/Posters/} \\
\texttt{/VT-Server/<department>/Projekte/<funding>/<project name>/Presentations/} \\
The directory structure is similar to that of papers, which means that you have
to save the used measured or simulated data as well. Some directories have of
course differing names and different content like, e.g.
\texttt{‘01\_Poster+Abstract’} which includes poster and abstract for the
conference. In case you have held an oral presentation, the first two
directories are named \texttt{‘00\_FinalPresentation’} and
\texttt{‘01\_Presentation+Abstract’}.
\begin{table}[!h]
  \caption{%
  The data of every poster has to be saved in seven subdirectories;
  further remarks: \\
  **When the poster was presented, the corresponding presenter loads the last
  and revised version (i.e. only the real used but absolutely complete data)
  into the protected directory \\
  ***If the data reproduced in a figure/table is from distributed raw data
  directories, it is sufficient to save here only the generating program%
  }
\begin{tabularx}{\linewidth}{l|p{1mm}X}
  \toprule
  \midrule
  \multirow{2}{*}[-17pt]{\texttt{00\_FinalPoster}$^*$} &
    \,\tabitem & Die allerletzte Version Ihres Posters als pdf mit Band, Jahr
                 und Seitenzahlen auf dem VT-Server in 'VT-Publikationen' \\
    & \,\tabitem & EndNote-Eintrag für IWT-WiKo mit DOI-Nummer, und ein EndNote
                   *.ris ASCII-Export, der den FAIR-Prinzipien entsprichts \\
  \midrule
  \texttt{01\_Poster+Abstract} &
    \,\tabitem & das *.pdf und *.pptx or *.tex \\
  \midrule
  \multirow{1}{*}[-25pt]{\texttt{02\_Figures}$^{**}$} &
    \,\tabitem & ein Unterverzeichnis für jede Abbildung (figure\_01, ...) mit
                 Datenpunkten als*.csv or *.txt, oder die erzeugende Datei/ das
                 erzeugende Programm für die Abbildungspunkte (Matlab, Python,
                 Origin,...); die Ergebnisse stammen aus Daten, die im
                 Verzeichnis \texttt{04\_PrimaryData} gespeichert oder daraus
                 extrahiert wurden \\
  \midrule
  \multirow{1}{*}[-7pt]{\texttt{03\_Tables}$^{**}$} &
    \,\tabitem & ein Unterverzeichnis für jede Tabelle (Tabelle\_01, ...) in dem
                 die Daten in Form einer csv-Datei gespeichert  \\
  \midrule
  \multirow{2}{*}[-8pt]{\texttt{04\_PrimaryData}} &
    \,\tabitem & gemessene oder simulierte Daten, jedes Unterverzeichnis mit
                 einer "readme.json" Datei zur Beschreibung der Daten \\
    & \,\tabitem & ELN pdf mit QR-code (siehe unten) \\
  \midrule
  \multirow{3}{*}{\texttt{05\_ProgramSources}} &
    \,\tabitem & e.g. OpenFOAM-Solver, Python/Matlab code, etc. \\
    & \,\tabitem & Fallbeispiele (Netze) for OpenFOAM, Fluent, ABAQUS etc. \\
    & \,\tabitem & “readme.json” Datei mit Metadaten \\
  \midrule
  \multirow{2}{*}{\texttt{06\_References}} &
    \,\tabitem & die Referenzen *.bib oder*.enl zusammen mit *.ris \\
    & \,\tabitem & Zitierte Arbeiten/Bücher (wenn Veröffentlichungsrechte
                   erlaubt sind) \\
  \midrule
  \multirow{1}{*}{\texttt{07\_Supplement}} &
    \,\tabitem & Ergänzende Dateien \\
  \midrule
  \multirow{1}{*}{\texttt{08\_Misc}} &
    \,\tabitem & Konferenzprogramm  \\
  \midrule
  \bottomrule
\end{tabularx}

\label{table:poster-directory-structure}
\end{table}
