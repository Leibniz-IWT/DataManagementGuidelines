\thispagestyle{empty}
\begin{center}
  \large\headerfont\fontseries{m}\textcolor{iwtdark}{Preface}
\end{center}

\begin{quote}
  `On July 20, 1969, Neil Armstrong climbed out of his spacecraft and
  placed his feet on the moon. The landing was broadcast live all over the
  world and was a significant event in both scientific and human History.
  Today, we can still watch the grainy video of the moon landing but what we
  cannot do is watch the original, higher quality footage or examine some of
  the data from this mission. This is because much of the data from early space
  exploration is lost forever. (...) The tapes were likely wiped and reused for
  data storage sometime in the 1970s.' \\
  \null\hfill - \citeauthor{briney2015}\cite{briney2015}
\end{quote}

\noindent This reader (https://github.com/Leibniz-IWT/DataManagementGuidelines) holds for all employees of the Leibniz-Institute for Materials Engineering (IWT) and serves as a reference
for how to save data. It delivers reasons for the question of why one should
think about data management and gives instructions and examples. Data management
in the information age is an absolutely required skill and a base to cope with
the wealth of data to avoid an information overload. It helps organizing
everyone’s own data, but it is also mandatory for data sharing so that others
can understand the structure and are able to interpret the information given by
your data. In this way, your data meet the FAIR standard (Findable, Accessible,
Interoperable and Reusable), basis for the looming new discipline of data science.

There are many topics in data management like data governance and security or
storage management, but we are treating here only a condensed extract, tailored
to data handling in our department. Therefore, this reader might help you to
clarify what must be done to be a data steward.
